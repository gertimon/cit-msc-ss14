\IEEEPARstart{S}{ince} the time computer were introduced in our society, there was also a necessity of file systems. At first only for saving programs or results of computations. Today we use file systems for nearly everything. We want not only to save multimedia files, such as pictures, music or videos, but programs and other files we work with too. Today nobody works without a file system on his or her \todo{reduce to his?} computer. The next progression of file systems came with the invention of networks and the internet. The amount of data which is created by e.g. a company or some other group of people, were too much and complicated to save it on a local file system in a computer. It was much easier to work and interact with these people over a distributed file system. These systems are powerful machines which are located somewhere in the world and can save a huge amount of data. Moreover distributed file systems provide a much better failure protection by mirroring discs or using systems like "Raid". Another advantage is, that such a system can be distributed over the whole world which gives everybody a nearly equal accessing delay. \todo{talk about file system "as a service", and the lie the cloud services would be low-priced as regular services maybe?}

In the 1990s these distributed file systems were mostly used for backups or saving and sharing files. Not just in the internet but in local networks of big companies where a lot of data accrues too. That means, these distributed systems where focused on bandwidth. The sooner files were up- or downloaded, the better was the file system and the network. Today we have some more requirements for distributed file systems. Of course bandwidth is still interesting, but also the "Quality of Service(QoS)". Today we would like to work directly on the files of the distributed system and not downloading them. If we watch a video stream from Youtube or talk with other people using Skype, we do not want to wait until the whole file is uploaded. It is much better if the stream starts directly by downloading successively small parts. In other words, today the latency is a very important feature, too. To accomplish this requirement, hardware and routing algorithms where improved. Another very promising idea, is the concept of "Software Definded Networking"(SDN), which is chosen for the system presented in this paper.

Today another problem is, that large distributed file systems can cost a huge amount of energy \todo{Hier noch Werte(Schon in Bearbeitung)}. Reducing this energy needs is not just important from the ecological point of view. But also for the organisers of the systems, who can save a lot of money, which means that consumer can save money, too. For example if a user needs to upload a big file in the next days. There is no reason not to upload the file over a long time without using 100 percent of the system, if the user can save energy with that. On the other hand there can be a situation, where a user needs a file immediately. Then it must be possible to get the file as fast as possible. But currently it is not trivial, that the user can decide that.  

Within this work, a file system was developed which combines all the requirements which were mentioned. This file system is organized and controlled by a "Software Defined Network". It is also monitored by the monitoring system "Zabbix", where the organizers can get a view into workload, network usage, current file transfers of users and energy costs. This system focuses on the energy problem. The users are able to use the system with so called "profiles". If a user chooses the profile "economical", he or she will be linked to a server, where the energy cost is low or where a lot of users are working in the moment, so they can share energy price. On the other hand, the profile "fast" will give the user as much power as possible. This can be realized by the SDN and some other parts, which will be described later. To have full control over the used energy, every user can check the own costs on a website for each hour. All in all this file system can be an example of how the next generation of distributed file systems can look like.               