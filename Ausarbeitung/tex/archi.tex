
\IEEEPARstart{B}{ased} on project restrictions and own decisions there was a hardware and software mixed test system created. This chapter will give a short introduction about how all the system parts are connected to each other. Detailed description about single parts will follow later.

\subsection{Hardware Setup}

Given by the project initiator, the institute for complex and distributed IT systems of TU Berlin (CIT) some hardware components were directly defined as prerequisites.

\begin{enumerate}

\item One powerful machine placed in tubit data centre (TU Berlin), later called by it's name \textit{Asok05}.

\item One energy-saving \textit{office pc} attached to TU network inside a staff office with lower bandwidth.

\item Two energy meters called \textit{EGM-PWM-LAN} of the brand \textit{Energenie} which were used to measure the power usage of \textit{Asok05} and the \textit{office pc}.

\item To run services which should not influence the power usage measured by the energy meters, a virtual machine, placed in the tubit data centre was prepared to be used too.

\end{enumerate}

\todo{add some machine details (storage/bandwidth/power/power usage idle)}

\subsection{Software Setup}

The introduced and following software systems were used by own decisions described later.

\begin{enumerate}

\item As distributed file system which controls the filesystem access the Apache Haddop\textsuperscript{\textregistered} Filesystem (HDFS) was used, described in chapter~\ref{sec:dfs}. The access to filesystem may be done using a terminal, http or a client as provided by Filesystem in Userspace (FUSE), a web client or terminal window. HDFS consists of two main parts, the namenode which organises the filesystem structural view, and datanodes which organise the stored amount of data in blocks.

\item To measure network traffic, all data were sent through a virtual network. A combination of Floodlight as controller and an Open vSwitch was used, described in chapter~\ref{sec:sdn}. 

\item For monitoring user actions in the filesystem and on the network we have used a zabbix server. this service collects data sent by an zabbix agent. \todo{chapter ref}

\item To analyze the changes in energy consumption produced by user activities the energy consumption of \textit{office pc} and \textit{Asok05} were measured. \todo{chapter ref?}

\item By evaluating the monitored data we were able to create a  report which informs about the power consumption used by all or a specified user. \todo{chapter ref}

\end{enumerate}

\subsection{Services Setup}

The distribution of services used in project are given in table~\ref{tab:services}.

\begin{table}[b]
	\centering
	\caption{Services and machines overview used in project. }
	\begin{tabular}{|l|l|l|l|l|}
		\hline \rule[-2ex]{0pt}{5.5ex} \textbf{client pc} & \textbf{office pc} & \textbf{virtual machine} & \textbf{Asok05} \\ 
		\hline \rule[-2ex]{0pt}{5.5ex} hadoop client & hadoop data node & hadoop name node & hadoop data node \\ 
		       \rule[-2ex]{0pt}{5.5ex}  & energy meter & zabbix server & energy meter \\ 
		       \rule[-2ex]{0pt}{5.5ex}  & zabbix agent & zabbix agent & zabbix agent \\ 
		       \rule[-2ex]{0pt}{5.5ex}  &  &  & Floodlight \\ 
		       \rule[-2ex]{0pt}{5.5ex}  &  &  & Open vSwitch \\ 
		       \rule[-2ex]{0pt}{5.5ex}  &  &  & energy data agent \\ 
		\hline 
	\end{tabular}
	\label{tab:services}
\end{table}
