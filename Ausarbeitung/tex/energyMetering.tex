\subsection{Energy Metering}

\subsubsection{Overview}
A main aspect of the project is the monitoring of energy consumption of the different machines. Interesting to know is how high are the peaks when a file system operation is proceeding. The Energy consumption of the two machines of the live system are monitored continuously by two energy meters. The data then is sent via a developed module every two seconds to the monitoring system Zabbix.

\subsubsection{Energy Meter}
The Energy Meter \textit{EGM-PWM-LAN} of the brand \textit{Energenie} was linked to the power supply unit of the machines. One to the bigger machine asok05 and the other one to the office computer. The metering device has an Ethernet interface for connecting it to the existing network. It is possible to connect to the device via an implemented web server which contains a web interface to perform configurations and to see current energy data. Furthermore the web interface can be reached via internet if it is connected to a router with internet access. Energenie offers a cloud service \cite{Energenie.2014} where the energy meter can be registered. Thereafter energy data is sent and diagrams are created by the cloud service. Generated diagrams illustrate the current power usage and the energy consumed in a certain time window. It is also possible to configure the electricity price divided in night- and daytime to show how much money the metered machine has cost.

\subsubsection{System Setup}
\dots


\subsubsection{Data Collection}
\dots
