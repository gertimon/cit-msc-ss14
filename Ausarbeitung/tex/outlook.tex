\label{sec:outlook}
In future, more smart ways of finding the best rack can be implemented. Algorithms which include parameters like energy prices, work load forecasts or historic user data are some ideas in which direction the development could go. Further the algorithm could consider different energy sources if a company got solar cells on the rooftop and on the basis of weather information it could switch to this energy source and run the machines completely free of charge. It could react to changing energy prices and ask the user whether he really wants to go on with its configured fast profile because at the moment it is quite expensive to operate in this profile.

Another Example is to use data from history. A user who never or rarely has had a fast connection to the smart file system do not need a fast profile. Its file operations is done with high energy consumption on the big machine but afterwards the link is like a bottleneck and the user feels that its operations are not very fast. The algorithm could learn over the time if there are data like geo-position or time of day available the user could be switched between the profiles automatically. Maybe the user lives in a village where the connection is poor that could be detected by the geo-position and could be automatically switched to the cheap profile. If a user is every day in his favourite cafe during lunch break and works from there it could be detected that the connection is poor during the lunch hours due to a lot of customers with whom the user has to share the line and could be switched automatically to the cheap profile.

The category of media should also be taken into consideration. Does a user just want to access a common file or download it the cheap profile is fair enough. If it is a stream what a user wants to start of course the data needs to be processed fast and using the fast rack is the better choice.
