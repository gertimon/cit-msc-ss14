\label{sec:outlook}

% --- \subsubsection{Future plans of dynamic selection}
% --- HDFS:


In future, more smart ways of finding the best rack can be implemented. Algorithms which include parameters like energy prices, work load forecasts or historic user data are some ideas in which direction the development could go. Further the algorithm could consider different energy sources if a company got solar cells on the rooftop and on the basis of weather information it could switch to this energy source and run the machines completely free of charge. It could react to changing energy prices and ask the user whether he really want to go on with its configured fast profile because at the moment it is quite expensive to operate in this profile.

An other Example is to use data from history. A user who never or rarely has had a fast connection to the smart file system do not need a fast profile. Its file operations is done with high energy consumption on the big machine but afterwards the link is like a bottleneck and the user feels that its operations are not very fast. The algorithm could learn over the time if there are data like geo-position or time available the user could be switched between the profiles automatically. Maybe the user lives in a village where the connection is poor that could be detected by the geo-position and could be automatically switched to the cheap profile.

The last Example is to use the service the user is using. If he only want to download, than a slow download can be possible. If he wants to stream a video we have to use a fast rack.

% --- SDN:




% --- Monitoring / Energy Metering